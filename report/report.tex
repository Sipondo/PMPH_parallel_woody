\documentclass[a4paper]{article}
\usepackage{pmph_proj}
\header{%
  assignment={Making Woody Parallel},%
  authors={Hugh McGrade <\texttt{wbr412@alumni.ku.dk}>\\
    Mads Obitsø <\texttt{scr411@alumni.ku.dk}>\\
    Titus Robroek <\texttt{robroek@di.ku.dk}>},%
  shortAuthors={\texttt{McGrade, Obitsø, Robroek}},%
  date={\today}
}

\allowdisplaybreaks
\begin{document}
\maketitle

\begin{abstract}
  
\end{abstract}

\section{Introduction}

\section{Motivation}

Something about remote sensing

\section{Woody Random Forest Prediction in Futhark}

\subsection{Interoperation of Woody and Futhark}

The Futhark compiler features code generation for the PyOpenCL library, allowing Futhark code compiled for OpenCL to be called from Python. This generated code accepts NumPy arrays as Futhark arrays. As Woody is written in Python, we are able to compile our Futhark library for PyOpenCL and use it as a library in Woody's Python code.

\section{Experimental Setup}

\section{Evaluation}

\section{Conclusion}

\end{document}
